% !TEX program = lualatex

% Use a Japanese-specific document class.
% The [lualatex] option is helpful for this class.
\documentclass[lualatex]{bxjsarticle}

% Load the package for Japanese typesetting with LuaLaTeX
\usepackage{luatex-ja}

% --- Your other packages can go here ---
\usepackage{amsmath}
\usepackage{hyperref}

\title{日本語の文書サンプル (Japanese Document Sample)}
\author{Your Name}
\date{\today}

\begin{document}

\maketitle

\section{はじめに (Introduction)}

これは日本語の文章をタイプセットするためのサンプルです。
This is a sample for typesetting Japanese text.

UTF-8エンコーディングが直接使えます。例えば、ここに漢字「日」やカタカナ「ラテフ」、ひらがな「こんにちは」を書くことができます。

You can mix it with English text and mathematical formulas seamlessly.
\[
    E = mc^2 \quad \text{そして} \quad \zeta(s) = \sum_{n=1}^{\infty} \frac{1}{n^s}
\]

こんにちは世界! (Hello, world!)

\end{document}
